\documentclass[12pt]{article}

%margins
\usepackage[margin=.5in,includefoot]{geometry}
% indenting
\usepackage{indentfirst}
\usepackage{float}
\usepackage{pdfpages}
\usepackage{multicol}
\setlength{\columnsep}{0.5cm}
\usepackage{mwe}
\usepackage{array}
\usepackage{graphicx}
\graphicspath{{graphs/}}
\usepackage{subcaption}
\usepackage[section]{placeins}
\usepackage{wrapfig}
\usepackage{setspace}

\begin{document}



	\begin{center}
	
	\textbf{\Large Thin Lens \& Mirror Experiment}\\
	[2mm]
	
	
	\textsc{ By Cameron Brooks, Raul Ballesteros, Mohammad Alharthy, Emanuel Villanuera,\\ Prince Rames, Ziggy Saravia, Mohammad Elsharrewy
}\\
	[2mm]
	\textsc{Dr. Derrick Kiley}\\
	\textsc{3/07/2018 }\\
	
	\end{center}





\begin{center}
\section*{Abstract}
\end{center} 
\noindent An experiment based the function and properties of thin lenses and mirrors was performed.An optical bench, light source, screen, short converging lens, long converging lens, diverging lens and a mirror were used in this experiment, to produce the appropriate distance from object to
lens, and distance from lens to screen. With the position of object, lens and screen be measured and recorded, the principle equations of focal length of lens can be applied. Meanwhile, using these equations which derived in the introduction section this theory is capable to be verified by making precise approximations, comparing with the actual values that can be measured and determined by a spherometer. The average error derived was less than 5 percent and comparing it to the standard deviation, the experiment could be concluded as a success.



\begin{multicols}{2}

\section*{Introduction}\label{sec:Intro}
A lens is a curved piece of glass which can focus light rays passing through it. The radii of curvature of the two sides of the lens are related to the focal length of the lens by equation ( 1) ( only if the lens is relatively thin):
\begin{equation}
\frac{1}{f} = (n-1)(\frac{a}{{R}_{1}}-\frac{a}{{R}_{2}})
\end{equation}
In this equation ( 1), ${f}$ is the value of the focal length, n is the index of refraction,${R}_{1}$ and ${R}_{2}$ are the radii of the curvature of two sides. ${R}_{1}$ and ${R}_{2}$ are both positive if the center of the curvature is on the opposite side of the lens from the incoming light rays. ${R}_{1}$ and ${R}_{2}$are negative if the center of curvature is on the same side of the incoming rays. Also, following the usual convention, the focal length ${f}$ can be approximated by equation (2) only when the lens is relatively thin, p represents the distance between the object and the lens, q represents the distance between the image and the lens:
\begin{equation} 
\frac{1}{f} = (\frac{1}{{p}}-\frac{1}{{q}})
\end{equation}
In addition, for a thin spherical mirror, the focal length ${f}$ can be calculated by equation (3):
\begin{equation} 
f = \frac{R}{2}
\end{equation}
Then, when two lenses are placed near each other the light from an object will be focused through both lenses. Let ${f_{1}}$, ${p_{1}}$ and ${q_{1}}$  represent the focal length, object distance and image distance relative to the first lens; ${f_{2}}$, ${p_{2}}$ and ${q_{2}}$ represent the focal length, object distance and image distance relative to the
second lens; L represents the distance between two lenses, then equation (4) shows:
\begin{equation} 
L={q}_{1}+{p}_{2}
\end{equation}
Finally, spherometer works by measuring the vertical distance between the center point of the lens or mirror and the center point of an equilateral triangle formed by the legs of the spherometer. The distance is called the sagitta s, and with distance between legs 1, the radius of curvature of a lens R can be calculated by equation (5):
\begin{equation}
R = \frac{s}{2}+\frac{{l}^{2}}{{6}_{s}}
\end{equation} 
In this experiment, the index of refraction of the lenses used is n= 1.53.

\section*{Procedures} 
For the first part of this experiment, we took a spherometer and calibrated it with fiat lens each time before measuring the curved lenses. This was done simply by placing the spherometer on a fiat piece of glass and turning the central screw around until it just touched the surface of the glass and recording the reading in millimeters. We placed a small piece of paper in between glass and spherometer to indicate just when the central screw touched the surface of the glass. It was important to note that each marking on the spherometer was a hundredth of a millimeter and that one rotation was equivalent to half a millimeter.

Next, we measured both sides of the four lenses using the same method as the calibration step and recorded their readings. Vie also measured the distance between each leg of the spherometer. which were aligned in an equilateral triangle, using Vernier calipers to help us later determine the radius of curvature of a lens. For the next part of our experiment, the apparatus was set up for us and we just placed our light source at the end of one side of our bench and clamped it in place at some distance, which we measured off of the markings from the bench itself. 

Then, we clamped our lenses and screen at a distance $p$ and $q$; $p$ being the object distance and $q$ the image distance. We did eight trials with the four lenses each time changing the image and the object distance until the image was focused on the screen. We did one trial were $p$ was greater than $q$ and the other trial vice versa. For the trials using the divergence lens, it was necessary to place the short converging lens close to the divergence lens so that we could obtain a sharp, focused image on the screen. In the case of the mirrors, we had to place the screen behind the light source for when $q > p$ and in between the source and the mirror for when $q < p$.
\par 

	\end{multicols}

\section*{Results and Calculations}\label{sec:RnD}
There were 7 sets of values measured by the spherometer (shown in Table 1) in order to calculate the radius of curvature on 4 different lenses that were used in this experiment. The distance between legs of the spherometer, 1, was also measured. Then, with the last equation in the introduction, every radius of curvature was also determined in below:

	

	\begin{table}[H]
\caption{Distance between legs of spherometer: 1 = 44.20mm}
\centering
	
\begin{tabular}[width=05\linewidth]{ccccc}
  	Lens or Mirror 	&Flat Reading	&Curved Reading	&Difference ( s)	&Radius (R)	\\
  	\hline
Short Converging Lens (side 1)	&0.280 mm &1.800 mm &1.520 mm &215.0 mm\\		
Short Converging Lens (side 1)	&0.271 mm &1.792 mm &1.521 mm &214.8 mm
\\
Long Converging Lens (side 1)	&0.242 mm &1.421 mm &1.179 mm &276.8 mm
\\	
Long Converging Lens (side 2)	&0.285 mm &1.472 mm &1.187 mm &274.9mm
\\	
Diverging Lens (side 1)	&0.280 mm &-0.810 mm &-1.090 mm &-299.3 mm\\	
Diverging Lens (side 2)	&0.276 mm &-0.819 mm &-1.095 mm &-297.9 mm\\	
Mirror	&0.280 mm &-0.879 mm &-1.159 mm &-281.5 mm\\	
      \hline
      \end{tabular}
\end{table}

In the 1st and the 2nd trials, a short converging lens was used, and a long converging lens was used in the 3rd and the 4th trials. And the positions of the object, lens and the screen were recorded shown in Table 2 (In the 1st and 3rd trials, $p$ was controlled shorter than $q$; in the 2nd and the 4th,$p$ was longer than $q$):
\begin{table}[H]
\caption{Experimental Data from Trials 1-4}
\centering
\begin{tabular}[width=05\linewidth]{ccccc}
     
Trial 	&Lens	&Position of object	&Position of lens	&Position of screen\\
\hline
1 &Short converging &12.5 cm &39.3 cm &108.6 cm\\
2 &Short converging &12.5 cm &79.7 cm &108.6 cm\\
3 &Long converging &12.6 cm &45.8 cm &145.3 cm\\
4 &Long converging &12.6 cm &109.8 cm &145.3 cm\\



      \hline
      \end{tabular}
\end{table}

Then, in the 5th and 6th trials, the diverging lens and the short converging lens were used together, in the fifth trial, the distance $p$ was controlled to be shorter than $q$, and sixth was on the contrary, all position values is shown in Table 3:
\begin{table}[H]
\caption{Experimental Data from Trials 5-6}
\centering
\begin{tabular}[width=05\linewidth]{ccccc}
     
Trial &Position of object &Position of DL &Position of CL &Position of screen\\
\hline
5 &12.5 cm &60.7 cm &98.2 cm &178.2 cm\\
6 &12.5 cm &67.4 cm &92.4 cm &133.2 cm\\
      \hline
      \end{tabular}
\end{table}

Finally, in the 7th and 8th trials, a mirror was used for test. The position values are shown in Table 4 (in the seventh, the $p$ was shorter than $q$, but the eighth was on the contrary):
\begin{table}[H]
\caption{Experimental Data from Trials 7-8}
\centering
\begin{tabular}[width=05\linewidth]{cccc}
     
Trial &Position of object &Position of mirror &Position of screen\\
\hline
7 &12.5 cm &36.4 cm &5.6 cm\\
8 &12.5 cm &45.3 cm &22.2 cm \\



      \hline
      \end{tabular}
\end{table}
With all this experimental data From Table 1 to Table 4, the calculation and analyze can be applied: First of all, with the radius of curvature values above, app lying the equation (1) or (3), the theoretical values of focal length of every lens or mirror can be determined and they are shown in Table 5:
\begin{table}[H]
\caption{Focal Length Theoretical Values}
\centering
\begin{tabular}[width=05\linewidth]{cc}
     
Type of Lens or Mirror     &Theoretical Focal Length ${f}_{theor}$\\
\hline
Short converging lens &20.27 cm\\
Long converging lens &26.02 cm\\
Diverging lens &-28.17 cm\\
Mirror &14.08 cm\\
      \hline
      \end{tabular}
\end{table}
Then, with the equation (2), the equation for determining the experimental values of focal length in trial 1, 2, 3, 4, 7 and 8 can be written as equation (6),(7) and (8) below:

\begin{equation}
	f = \frac{p \cdot q}{p+q}
	\end{equation} 
	\begin{equation}
	p = {position}_{(lens)} - {position}_{(object)}
	\end{equation} 
\begin{equation}
	q = {position}_{(screen)} - {position}_{(lens)}
	\end{equation} 
Therefore, the distance of $p$, $q$ and experimental values of focal length for each trial can be determined and they are shown in Table 6:
\begin{table}[H]
\caption{Focal Length Experimental Values 1}
\centering
\begin{tabular}[width=05\linewidth]{cccc}
     
Trial &$p$ &$q$ &Experimental Focal Length (${f}_{exp}$)\\
\hline
1 (SC)&26.8 cm&69.3 cm&19.33 cm\\
2 (SC)&67.2 cm&28.9 cm&20.21 cm\\
3 (LC)&33.2 cm&99.5 cm&24.89 cm\\
4 (LC)&97.2 cm&35.5 cm&26.00 cm\\
7 (M)&23.9 cm&30.8 cm&13.46 cm\\
8 (M)&32.8 cm&23.1 cm&13.55 cm \\

      \hline
      \end{tabular}
\end{table}

In the 5th and 6th trials, the short converging lens was second lens the light pass through. Therefore, the distance between converging lens and the screen ${q}_{2}$ and the focal length of the converging lens ${f}_{2}$ were known, with the equation (9), ${p}_{2}$ can be determined:
\begin{equation}
{p}_{2} = \frac{{f}_{2} \times  {q}_{2}}{{f}_{2} -  {q}_{2}}
\end{equation}

Then, with the value of ${p}_{2}$, and the distance between both lenses L, the value of ${q}_{1}$ can be found by
the equation (10):
\begin{equation}
{q}_{1} = L - {p}_{2}
\end{equation}
Now, the distance between the diverging lens and object $p$ and the distance between object and screen $q$ were known, the experimental focal length of the diverging length can be solved, and Table 7 the data and results from these two trials:
\begin{table}[H]
\caption{Focal Length Experimenta l Values 2}
\centering
\begin{tabular}[width=05\linewidth]{cccccc}
     
Trial &${p}_{1}$&L &${q}_{2}$ &${q}_{1}$ &${f}_{1}$\\
\hline
5 &48.2 cm &37.5 cm &80.0 cm &64.6 cm&-27.6 cm\\
6 &54.9 cm &25.0 cm &40.8 cm &65.3 cm&-29.8 cm\\
      \hline
      \end{tabular}
\end{table}
Since the experimental local length values have all been determined, now they can be compared with the theoretical values by using the equation (11) to solve the percentage of errors and prove the theories:
\begin{equation}
	error \% = \frac{|{f}_{exp} - {f}_{theor}| \cdot 100 \%}{{f}_{theor}}
	\end{equation} 

Finally, all percentages of errors from all the trials are calculated and they are shown in Table 8:

\begin{table}[H]
\caption{Percentage of Error}
\centering
\begin{tabular}[width=05\linewidth]{cccc}
     
Trial &Theoretical focal length &Experimental Focal Length &Error\% \\
\hline
1 &20.27 cm &19.33 cm &4.64\% \\
2 &20.27 cm &20.21 cm &0.30\%\\
3 &26.02 cm &24.89 cm &4.34\% \\
4 &26.02 cm &26.00 cm &0.08\% \\
5 &-28.17 cm &-27.6 cm &2.02\% \\
6 &-28.17 cm &-29.8 cm &5.79\% \\
7 &14.08 cm &13.46 cm &4.40\% \\
8 &14.08 cm &13.55 cm &3.76\% \\
      \hline
      \end{tabular}
\end{table}

Therefore, the percentage of error from the experimental values in the double slit experiment are mostly below $5\%$ comparing with the theoretical data.

\begin{multicols}{2}

\section*{Conclusion}
This lab was targeting to get a better understanding on the function and properties of thin lenses and mirrors. The results that were found in the experiment were very accurate that most of the  experimental results ended up with less than 5\% of the percentage error comparing to the theoretical values and confirmed that the principles that were derived in the introduction section are very accurate at making prediction on the properties of thin lenses and mirrors. 
However, some errors were acquired in our experiment. Possible sources of error include the precision of the measuring equipment and taking the measurements. Overall, the experiment was a success.

\section*{REFERENCES}
Los Angeles City College Lab Manual Physics 103. 

\end{multicols}
\end{document}

